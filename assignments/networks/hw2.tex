\documentclass[14pt]{article}
\usepackage{url}
\usepackage{hyperref}
\usepackage{amsmath}
\usepackage{mathtools}
\usepackage{extsizes}
\usepackage{titling}
\usepackage{siunitx}
\usepackage{graphicx}
\usepackage[shortlabels]{enumitem}
\usepackage[margin=0.75in]{geometry}

\newcommand{\bd}{\textbf}

\setlength{\droptitle}{-5em} 

\title{Computer Networks HW2}
\author{Mitchell Meier}
\date{\today}

\begin{document}

\maketitle

\begin{enumerate}

\item
\begin{enumerate}[(1)]
\item
The minimum time needed to distribute this file from a server to all clients in a client-server model is $D_{cs}$  

\[D_{cs} = max(\frac{NF}{u_s},\frac{F}{d_{min}})\]

With $N$ equal to the number of clients (7), $F$ is equal to the file size (4 Gigabits), $u_s$ is equal to the upload rate of the server (94 Megabits per second), and $d_{min}$ is equal to the lowest download rate out of all clients ($c_1$ in this case, 15 Megabits per second). Using these numerical values, the solution to $D_{cs}$ can be simplified to

\[D_{cs} = max(297.87,266.67)\]

Which means the minimum time needed to distrubute F to all clients will be 297.87 seconds \\

\item
The server, s, is the cause for the final minimum time. The server is guaranteed to take longer to upload the file F to each individual client than the slowest client is to download that file \\

\item
The minimum time needed to distribute this file from a server to all clients in a peer to peer model is $D_{P2P}$

\[D_{P2P} = max(\frac{F}{u_s},\frac{F}{d_{min}},\frac{NF}{u_s + \sum\limits_{i=1}^{N} u_i})\] 

With $N$ equal to the number of clients (7), $F$ is equal to the file size (4 Gigabits), $u_s$ is equal to the upload rate of the server (94 Megabits per second), $d_{min}$ is equal to the lowest download rate out of all clients ($c_1$ in this case, 15 Megabits per second), and $\sum_{i=1}^{N} u_i$ is equal to the combined upload bandwith of all clients in the system (135 Megabits per seond). Using these numerical values, the solution to $D_{P2P}$ can be simplified to 

\[D_{P2P} = max(42.55,266.67,122.27)\] 

Which means the minimum time needed to distrubute F to all clients will be 266.67 seconds \\

\item
The client, $c_1$, is the cause for the final minimum time. The client is guaranteed (assuming all clients are utilizing their full upload speed) to take longer to download the file F than it will take the other clients to upload/share the file with each other \\
\end{enumerate}

\item
Yes, it is possible for an application to utilize reliable data transfer even while not using TCP. An application developer can use UDP and then implement reliable data transfer into the application layer. A developer may do this to avoid some of TCP's built in features, such as congestion control \\

\item
Each connection socket on the Web Server Host C is identified with a four tuple including source(client) IP address, source port number, destination port number, and destination IP address. Since the requests from A and B are distinguishable from each other since they have different source IP addresses, the web server will be able to designate each connection to a seperate socket (they still have the same port number however). This would not work with UDP however, since UDP does not include source IP address

\item
Sequence numbers help the reciever categorize arriving packets and determine if they contain new or retransmitted data. Sequence numbers also support reordering and provide insight on potential dropped packets \\

\item
Timers help detect lost packets. By detecting if the ACK for a transmitted packet is not recieved within the time specified by the timer, the packet is projected to have been lost, and the packet is transmitted again \\

\item
\begin{enumerate}[(a)]
\item
False, Host B will still send acknowledgements to Host A

\item
False, rwnd is dynamic

\item
True

\item
False, the sequence number depends on the bumber of 8 byte characters in the current segment

\item
True

\end{enumerate}
\item
\begin{enumerate}[(1)]
\item
$0010000000111101$

\item
$1101111111000010$ 

\end{enumerate}

\item
\begin{enumerate}[(1)]
\item
Segment's sequence number at $t=1$ is 118
Segment's sequence number at $t=2$ is 764
Segment's sequence number at $t=3$ is 1410
Segment's sequence number at $t=4$ is 2056

\item
ACK's sequence number sent when $t=8$ is 764
ACK's sequence number sent when $t=9$ is 1410
ACK's sequence number sent when $t=10$ is 2056
ACK's sequence number sent when $t=11$ is 'x'

\end{enumerate}

\item
RTT propgation delay between hosts is 30ms (Double the given propgation delay) \\
Transmission rate = $10^9 bps = R$ \\
Size of data packet = $1500 * 8 = 12000 bits = L$ \\

$D_{trans} = L/R$ \\
$D_{trans} = 0.000012 s$ 

$Channel Utilization = N * \frac{12micros}{12micros + RTT}$ \\
$98/100 = N * \frac{12ms}{30.012ms}$ \\
$N = 2450.98$

For 98 percent utilization, window size should be approximately 2451 packets long

\end{enumerate}

\end{document}