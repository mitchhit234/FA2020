\documentclass[14pt]{article}
\usepackage{url}
\usepackage{hyperref}
\usepackage{amsmath}
\usepackage{mathtools}
\usepackage{extsizes}
\usepackage{titling}
\usepackage{siunitx}
\usepackage{graphicx}
\usepackage[shortlabels]{enumitem}
\usepackage[margin=0.75in]{geometry}
\usepackage{indentfirst}
\usepackage{caption}

\newcommand{\bd}{\textbf}

\setlength{\droptitle}{-5em} 

\title{Computer Networks Homework 4}
\author{Mitchell Meier}
\date{\today}

\begin{document}

\maketitle

\begin{enumerate}

\item
Bit Generator $G = 1001$ \\
Data Payload $D = 10011101$ \\
Size of R, $r = 3$ \\

For R to be valid, the following must be true: \\

$(D + R) \% G = 0$ \\
$(10011101 + R) \% 1001 = 0$ \\
$R =$ Remainder of $\frac{D*2^r}{G}$ \\

Steps of $1001 \div 10011101000$ : \\
Compare first 4 bits 1001 XOR 1001 = 0000, move onto next 4 bits \\
1101 XOR 1001 = 0100, bring down 1 bit \\
1000 XOR 1001 = 0001, bring down last 2 bits \\
000100 = 100 = R


\item
Frames have L bits, R transmission speed \\
1 slot takes L/R time to complete, 3 active nodes \\

To find the efficeny of slotted ALOHA, we need to find the percent of total time where one node is transmitting and the rest are not \\

$p =$ probability a given node is transmitting \\
$1-p$ is the probability a given node is not transmitting \\
$p(1-p)^{N-1}$ is the probabily a given node is transmitting and the rest are not \\
$Np(1-p)^{N-1}$ is the probability that any of the nodes are transmitting and the rest are not (the case in which no collisions occur and a slot is successful in sloted ALOHA)\\

To find the efficeny of regular ALOHA, we need to find the percent of total time where one node is transmitting a frame and no other node begins a transmission of a frame during that time (we assumed the same L/R time to transmit a frame) \\

$p(1-p)^{2(N-1)}$ now represent the probability that a given node is transmitting and the rest are not \\
$Np(1-p)^{2(N-1)}$ is the probability that any of the nodes are transmitting and the rest are not

\begin{enumerate}[(1)]
\item 
Slotted ALOHA - $ME = 3(0.37)(1-0.37)^2 = 0.44$ R bits per second \\
Normal ALOHA - $ME = 3(0.37)(1-0.37)^4 = 0.17$ R bits per second

\item
Slotted ALOHA - $ME = 3(0.59)(1-0.59)^2 = 0.298$ R bits per second \\
Normal ALOHA - $ME = 3(0.59)(1-0.59)^4 = 0.05$ R bits per second
\end{enumerate}

\item
Encoded channel output = all $Z_{i,m} = d_i \times c_{m}$ for i in M, and the m bit
\begin{enumerate}[(1)]
\item
The encoded output for bit 1 is the same as M, -1,-1,-1,-1,1,-1,-1,-1 
\item
The decoded bit value for the channel output 1,1,1,1,-1,1,1,1 is -1

\end{enumerate}

\item
No, because TCP's reliable delivery service could be utilized in network topoligies beside the internet's, where that network's lower layers may or may not provide reliable delivery service

\item
Range of K = ${0,1,2,...2^{n-1}}$ \\
$n = 5$ , Range of K = [0, 16] \\
Probability K = 4 is $\frac{1}{17}$

\item
There would be no advantage because increasing the size of RTS and CTS frames would also increase the probability of a collision, and that collision would last the same amout of time as it would for the same length DATA and ACK frames

\item
The user and node are not consideremultipule different access points when changing locations and connecting to a network. Since the laptop is always accessing a network through the same access point, it is not considered mobile
\end{enumerate}

\end{document}